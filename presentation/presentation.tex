\documentclass{beamer}

%\usepackage{pgf}
%\usepackage[autosize,pgf]{dot2texi}
%\usepackage{tikz}
\usepackage{mystyle}
\renewcommand \o \overline

\setbeamersize{text margin left=15pt, text margin right=15pt}

\setbeamercovered{transparent}

\begin{document}
\begin{frame}
  Descriptive view of everything:
  \begin{itemize}
  \item Graphs are on the vocabulary
    \begin{equation*}
      \tau_g \define \bigl\langle E^2 \bigr\rangle
    \end{equation*}
    and are required to be symmetric.
    \pause
    \item We want to define decomposition schemes as FO[ITER] queries
      from $\tau_g$ to some vocabulary of treelike decompositions.
  \end{itemize}
\end{frame}


\begin{frame}
  ``Vocabularies for tuple structures''
  \begin{itemize}
  \item We've seen several constructions of tree[like] decompositions where
    each node is constructed from a fixed-arity tuple of vertices.
  \item Our objects of interest (bags, separators, components, cones) are
    sets of vertices, one at each node.
    \pause
  \item Awkward: make these binary relations on nodes?  Tree[like]
    decompositions are defined on the vocabulary $\bigl\langle \sigma^2,
      \alpha^2 \bigr\rangle$
  \end{itemize}
\end{frame}

\begin{frame}
  Better solution (and the one Grohe uses):
  \begin{itemize}
  \item Query onto a decomposition has \emph{dimension} $d$ (arity).
  \item Nodes are $d$-tuples of vertices.
  \item Node variables written $\overline{x}$, vertex variables written
    $x$
    \pause
  \item Vocabulary of decompositions:
    \begin{equation*}
      \tau_{d} \define \bigl\langle E(\overline{x}, \overline{y}),
        \sigma(\overline{x}, y), \alpha(\overline{x}, y) \bigr\rangle
    \end{equation*}
    \pause
  \item A \emph{decomposition scheme} (i.e. FO[IFP] query) from $\tau_{g}$
    to $\tau_d$ is then:
    \begin{equation*}
      \Lambda = \bigl(\lambda_V(\overline{x}), \lambda_E(\overline{x},
      \overline{y}), \lambda_\sigma(\overline{x}, y),
      \lambda_\alpha(\overline{x}, y)\bigr)
    \end{equation*}
    \begin{itemize}
    \item Recall $\lambda_V(\overline{x})$ defines the nodes/universe.
    \end{itemize}
  \end{itemize}
\end{frame}

\begin{frame}
  \begin{equation*}
    \Lambda = \bigl(\lambda_V(\overline{x}), \lambda_E(\overline{x},
    \overline{y}), \lambda_\sigma(\overline{x}, y),
    \lambda_\alpha(\overline{x}, y)\bigr)
  \end{equation*}
  \begin{equation*}
    \tau_{d} \define \bigl\langle E(\overline{x}, \overline{y}),
      \sigma(\overline{x}, y), \alpha(\overline{x}, y) \bigr\rangle
  \end{equation*}
  \begin{equation*}
    \tau_{g} \define \bigl\langle E^2 \bigr\rangle
  \end{equation*}
  \begin{itemize}
  \item $\lambda$ formulas in FO[IFP].
    \pause
  \item $\Lambda : \text{STRUCT}[\tau_g] \to \text{STRUCT}[\tau_d]$
    \pause
  \item $\Lambda : \text{WFF}[\tau_d] \to \text{WFF}[\tau_g]$. (dual)
    \pause
  \item   Integrity Constraints:
    \begin{itemize}
    \item $\tau_g$ contains undirected graphs
    \item $\tau_d$ contains DAGs
    \end{itemize}
  \end{itemize}
\end{frame}

\begin{frame}
  Bag and Cone are FO definable:
  \begin{itemize}
  \item We have
    \begin{equation*}
      \tau_{d} \define \bigl\langle E(\overline{x}, \overline{y}),
      \sigma(\overline{x}, y), \alpha(\overline{x}, y) \bigr\rangle
    \end{equation*}
  \item Cone:
    \begin{equation*}
      \gamma(\overline{t}, x) \define \sigma(\overline{t}, x) \Or
      \alpha(\overline{t}, x)
    \end{equation*}
  \item Bag:
    \begin{align*}
      \beta(\overline{t}, x) &\define \gamma(\overline{t}, x) \And \Not
      \exists \overline{u}\ E(\overline{t}, \overline{u}) \And
      \alpha(\overline{u}, x)
    \end{align*}
  \end{itemize}
\end{frame}
% % \begin{dot2tex}
% % \end{dot2tex}

\begin{frame}
  Tree decompositions are FO[IFP] definable:
  \begin{equation*}
    \tau_{d} \define \bigl\langle E(\overline{x}, \overline{y}),
    \sigma(\overline{x}, y), \alpha(\overline{x}, y) \bigr\rangle
  \end{equation*}
  \begin{itemize}
  \item \textbf{(T.0)}: $E$ forms a directed tree.
  \item \textbf{T.1}: For all $x$, the set $\{ \overline{t} \st
    \beta(\overline{t}, x)\}$ forms a subtree under $E$.
  \item \textbf{T.2}: For all $x$, $y$ with $E(x, y)$ (in original graph)
    there exists $\overline{t}$ s.t. $\beta(\overline{t}, x) \And
    \beta(\overline{t}, y)$.
  \end{itemize}
\end{frame}

\begin{frame}
  Double-smile equivalence: FO-definable.
  $\overline{x} \doublesmile \overline{y} \define \sigma(\overline{x}, -) =
  \sigma(\overline{y}, -) \And \alpha(\overline{x}, -) =
  \alpha(\overline{z}, -)$

  or equivalently

  $\overline{x} \doublesmile \overline{y} \define (\forall z)\  \sigma(\overline{x}, z) \Iff
  \sigma(\overline{y}, z) \And \alpha(\overline{x}, z) \Iff
  \alpha(\overline{y}, z)$  
\end{frame}

\begin{frame}
  Treelike decompositions are FO[IFP] definable:
  \begin{equation*}
    \tau_{d} \define \bigl\langle E(\overline{x}, \overline{y}),
    \sigma(\overline{x}, y), \alpha(\overline{x}, y) \bigr\rangle
  \end{equation*}
  \begin{itemize}
  \item \textbf{TL.1}: $E$ is acyclic.
  \item \textbf{TL.2a}: $\forall \overline{t} \Not \exists x\ 
    \alpha(\overline{t}, x) \And \sigma(\overline{t}, x)$
  \item \textbf{TL.2b}: $\forall \overline{t} \forall x y\ 
    \alpha(\overline{t}, x) \And E(x, y) \Implies \sigma(\overline{t}, y)$
    (``separators separate'')
   \item \textbf{TL.3}: $\forall \overline{t} \forall \overline{u}
     . E(\overline{t}, \overline{u}) \forall x \bigl(\alpha(\o{t}, x)
     \Implies \alpha(\o{u}, x)\bigr) \And \bigl(\sigma(\o{t}, x) \Implies
     \sigma(\o{u}, x)\bigr)$
   \item \textbf{TL.4}: 
     \vspace{-5mm}
     \begin{align*}
       \forall \overline{t} &\forall \o{u_1} . E(\o{t},
       \o{u_1}) \forall \o{u_2} . E(\o{t}, \o{u_2}) \\
       &{} \forall x \ \bigl(\gamma(\o{u_1}, x) \And \gamma(\o{u_2},
       x)\bigr) \Iff \bigl( \sigma(\o{u_1}, x) \And \sigma(\o{u_2}, x) \bigr) \\
     \end{align*}
     \vspace{-10mm}
   \item \textbf{TL.5}: For every connected component $A$ of the original
     graph $G$, there is some $\overline{t}$ with $\sigma(t) = \nullset$
     and $\alpha(t) = V(A)$.
  \end{itemize}
\end{frame}

\begin{frame}
  \frametitle{Torsos}
  \begin{itemize}
  \item Defined as a graph-valued function:
    \begin{equation*}
      \tau(\overline{t}) \define G[\beta(\overline{t})] \union
      K[\sigma(\overline{t})] \union \Union_{\o{u} \in N_+(\o{t})} K[\sigma(\o{u})]
    \end{equation*}
  \item Since the nodes are exactly the bag, I won't define the node
    relation again.
  \item The torso's edge relation is clearly $\FO$-definable
    \begin{align*}
      \tau(\overline{t}, x, y) &\define \beta(\o{t}, x) \And \beta(\o{t},
      y) \And {} \\
      \Bigl[ E(x, y&) \Or \bigl( \sigma(\o{t}, x)
      \And \sigma(\o{t}, y) \bigr) \Or \bigl( \exists \overline{u} . E(\overline{t}, \overline{u}) \And
      \sigma(\o{u}, x) \And \sigma(\o{u}, y) \bigr) \Bigr]
    \end{align*}
  \item{} [Switch to other presentation]
  \end{itemize}
\end{frame}

\begin{frame}
\setbeamercovered{invisible}
  \begin{columns}
    \column{.5\textwidth}
    \includegraphics[scale=0.5]{fig1.pdf}

    Example graph.

    \column{.5\textwidth}
    \pause
    \includegraphics[scale=0.6]{fig2.pdf}

    Tree decomposition (bags).
    \pause

    But we'd rather work with separators.
  \end{columns}
\end{frame}

\begin{frame}
  \begin{columns}
    \column{.5\textwidth}
    \includegraphics[scale=0.5]{fig1.pdf}
    \column{.5\textwidth}
    \includegraphics[scale=0.6]{fig3.pdf}

    Separators (left) and Components (right).
  \end{columns}
\end{frame}

\begin{frame}
  \begin{columns}
    \column{.5\textwidth}
    \includegraphics<1>[scale=0.5]{fig1b.pdf}
    \includegraphics<2>[scale=0.5]{fig1c.pdf}
    \includegraphics<3>[scale=0.5]{fig1d.pdf}
    \includegraphics<4>[scale=0.5]{fig1e.pdf}
    \column{.5\textwidth}
    \includegraphics<1>[scale=0.6]{fig3b.pdf}
    \includegraphics<2>[scale=0.6]{fig3c.pdf}
    \includegraphics<3>[scale=0.6]{fig3d.pdf}
    \includegraphics<4>[scale=0.6]{fig3e.pdf}
  \end{columns}
\end{frame}

\begin{frame}
  % \frametitle{$\tau(\overline{t}) \define G[\beta(\overline{t})] \union
  %     K[\sigma(\overline{t})] \union \Union_{\o{u} \in N_+(\o{t})}
  %     K[\sigma(\o{u})]$}
  \begin{equation*}
    \tau(\overline{t}) \define G[\beta(\overline{t})] \union
    K[\sigma(\overline{t})] \union \Union_{\o{u} \in N_+(\o{t})}
    K[\sigma(\o{u})]
    \end{equation*}

  \begin{columns}
    \column{.5\textwidth}
    \includegraphics<1>[scale=0.5]{fig1b2.pdf}
    \includegraphics<2>[scale=0.5]{fig1c2.pdf}
    \includegraphics<3>[scale=0.5]{fig1d2.pdf}
    \includegraphics<4>[scale=0.5]{fig1e2.pdf}
    \column{.5\textwidth}
    \includegraphics<1>[scale=0.6]{fig3b.pdf}
    \includegraphics<2>[scale=0.6]{fig3c.pdf}
    \includegraphics<3>[scale=0.6]{fig3d.pdf}
    \includegraphics<4>[scale=0.6]{fig3e.pdf}
  \end{columns}
\end{frame}

\begin{frame}
  \frametitle{So what are torsos?}
  \begin{itemize}
  \item We can categorize graphs by whether you can express them with a
    particular set of torsos.
  \item A treelike decomposition is \emph{over} a class of graphs
    $\mathcal{A}$ if all its torsos are from $\mathcal{A}$.
  \item We have some idea of what graphs of bounded treewidth are.
    \begin{itemize}
    \item I think Neil said it's the same as number of variables, which
      mostly makes sense to me.
    \end{itemize}
  \item Graphs with treewidth $k$ are exactly the graphs that can be drawn
    with $k+1$-vertex torsos (since the vertices of the torsos are exactly
    the bags).
    \begin{itemize}
    \item Forests $\mathcal{T}_1 = \mathcal{T}(\mathcal{G}_2)$
    \end{itemize}
  \item Definability of torsos related to definability of graph.
  \end{itemize}
\end{frame}

\begin{frame}
  \frametitle{Let's talk about definable decompositions again}
  \vspace{-5mm}
  \begin{equation*}
    \Lambda = \bigl(\lambda_V(\overline{x}), \lambda_E(\overline{x},
    \overline{y}), \lambda_\sigma(\overline{x}, y),
    \lambda_\alpha(\overline{x}, y)\bigr)
  \end{equation*}
  \begin{itemize}
  \item We don't require $\Lambda$ to define a treelike decomp.\ for every
    graph $G$.  \pause
  \item $\mathcal{T}_{\Lambda}$ is the class of all graphs $G$
    s.t. $\Lambda[G]$ is a treelike decomposition.
    \begin{itemize}
    \item Seemed kind of backwards at first.
    \item Pre-image of TL.1-TL.5 under $\Lambda$.
    \item The goal of a decomposition scheme is to map (``decompose'') a
      bunch of graphs onto treelike decompositions.
      \pause
    \item Note $\mathcal{T}_{\Lambda}$ is FO[IFP]-definable.
    \end{itemize}
    \pause
  \item $\mathcal{T}_{\Lambda}[\mathcal{A}]$ -- class of graphs where
    $\Lambda$ is a treelike decomp.\ over $\mathcal{A}$ (torsos).
    \begin{itemize}
    \item Typically (always?) $\mathcal{A}$ will be FO[IFP]-definable.
    \item In addition to TL.1-TL.5, restrictions on torso (bag, separator,
      children's separators).
      \pause
    \item $\mathcal{T}_{\Lambda}[\mathcal{A}]$ is also FO[IFP]-definable!
    \end{itemize}
  \end{itemize}
\end{frame}

\begin{frame}
  \frametitle{Lemma 5.4.3 -- Definability Lifting Lemma}
  \begin{itemize}
  \item If $\Lambda$ and $\mathcal{A}$ are IFP-definable, then so is
    $\mathcal{T}_{\Lambda}[\mathcal{A}]$
    \begin{itemize}
    \item Recall $\mathcal{T}_{\Lambda}[\mathcal{A}]$ -- class of graphs where
    $\Lambda$ is treelike decomp.\ over $\mathcal{A}$.
    \end{itemize}
  \item Given a TL decomp.\ we can calculate its torsos, and check each of
    them for membership in $\mathcal{A}$.
  \item FO[IFP]-definable sets are closed under pre-image of
    FO[IFP]-definable functions. (``syntactic interpretation lemma'').
  \end{itemize}
\end{frame}

\begin{frame}
  \frametitle{Lemma 5.4.5 -- Union Lemma for Definable Decompositions}
  
  For $\mathcal{A}$ IFP-definable and $\mathcal{B}$, $\mathcal{C}$ admitting
  IFP-definable treelike decompositions over $\mathcal{A}$, we have that
  $\mathcal{B} \Union \mathcal{C}$ admits an IFP-definable treelike
  decomposition over $\mathcal{A}$.

  \begin{itemize}
  \item Assume wlog both decompositions have the same dimension.
  \item There are schemes $\Lambda^B$ and $\Lambda^C$ s.t. $\mathcal{B}
    \subseteq \mathcal{T}_{\Lambda^B}(\mathcal{A})$ and $\mathcal{C}
    \subseteq \mathcal{T}_{\Lambda^C}(\mathcal{A})$
  \item There is an IFP formula $\phi_B$ defining the class
    $\mathcal{T}_{\Lambda^B}(\mathcal{A})$.
  \item Our decomposition basically uses $\Lambda^B$ if it works, otherwise
    uses $\Lambda^C$
    \begin{equation*}
      \lambda_\sigma(\o{x}, y) \define \bigl(\phi_B \And
      \lambda_\sigma^B(\o{x}, y)\bigr) \Or \bigl( \Not \phi_B \And
      \lambda_\sigma^C(\o{x}, y)\bigr)
    \end{equation*}
  \end{itemize}


\end{frame}

\begin{frame}
  \frametitle{Treelike decomposition examples}
  \begin{columns}
    \column{0.3\textwidth}
    \includegraphics[scale=0.4]{fig6.pdf}
    \column{0.7\textwidth}
    \includegraphics<1>[scale=0.6]{fig6b.pdf}
    \includegraphics<2>[scale=0.6]{fig6c.pdf}
  \end{columns}
\end{frame}

\begin{frame}
  \setbeamercovered{invisible}
  \frametitle{\emph{Normal} Treelike Decompositions}
  \vspace{-8mm}
  \begin{columns}
    \column{0.3\textwidth}
    \includegraphics[scale=0.4]{fig6.pdf}
    \column{0.7\textwidth}
    \includegraphics[scale=0.6]{fig6c2.pdf}
  \end{columns}  
  \vspace{-10mm}
    \only<2> {
      \begin{itemize}
      \item (Green) Every source has $\sigma = \nullset$ and $\alpha = $ the
        vertex set of some connected component of $G$.
      \item Note (TL.2) required every connected component to own some such
        node.
        \end{itemize}
    }
    \only<3> {
    (Red) No nodes with $\gamma = \nullset$.
    }
    \only<4> {
    Note FO[IFP]-definable.
    }
\end{frame}

\begin{frame}
  \frametitle{Normal Treelike Decompositions}
  \begin{itemize}
  \item (Lemma 4.3.5): We can turn any TL decomp.\ normal by recursively
    pruning out ``bad'' nodes.
  \item After this process, all of the remaining nodes ``look exactly the
    same as they used to''.
    \begin{itemize}
    \item Given two treelike decompositions $\Delta, \Delta'$ of the same
      graph and nodes $\overline{t} \in \Delta$, $\overline{t'} \in
      \Delta'$, we write $\overline{t} \triplesmile \overline{t'}$ if they
      have the same bag and component and cone and design and torso.
    \item Nodes are ``$\Delta, \Delta'$-indistinguishable'' or
      ``indistinguishable'' or ``equivalent''.
    \item Note $\overline{t} \triplesmile \overline{t'} \Implies
      \overline{t} \doublesmile \o{t'}$ but not vice versa.
    \end{itemize}
  \item (Lemma 5.2.3): Normalization process is FO[IFP]-definable.
  \end{itemize}
\end{frame}

\begin{frame}
  \frametitle{\emph{Strict} Treelike Decompositions}
  \vspace{-5mm}
  \begin{columns}
    \column{0.3\textwidth}
    \includegraphics[scale=0.4]{fig6.pdf}
    \column{0.7\textwidth}
    \includegraphics<1>[scale=0.6]{fig6c.pdf}
    \includegraphics<2>[scale=0.6]{fig6c3.pdf}
  \end{columns}  
  \vspace{-10mm}
  \begin{itemize}
  \item Every arrow gets rid of something or moves something from $\alpha$
    to $\sigma$.
  \end{itemize}
\end{frame}

\begin{frame}
  \frametitle{Strict Treelike Decompositions}
  \begin{itemize}
  \item Lemma 4.3.6: We can turn a decomp.\ $\Delta$ into a strict decomp.\
    $\Delta'$ s.t.:
    \begin{itemize}
    \item $V(\Delta') \subseteq V(\Delta)$ 
    \item $\o{t} \unlhd^{\Delta'} \o{u} \Iff \o{t} \unlhd^{\Delta} \o{u}$
    \item $\forall \o{t} \in V(\Delta')\ \ \o{t} \triplesmile^{\Delta', \Delta} \o{t}$
    \end{itemize}
  \item Lemma 5.2.3: Strictifying process is FO[IFP]-definable.
  \item Lemma 4.3.7: We can make a graph strict-and-normal in FO[IFP],
    because normalizing process from Lemma 4.3.5 preserves strictness.
  \end{itemize}
\end{frame}

\begin{frame}
  \frametitle{\emph{Tight} Decompositions}
  \begin{itemize}
  \item A decomposition is \emph{tight} if, for all $\overline{t}$,
    \begin{itemize}
    \item $\alpha(\overline{t})$ is connected
    \item $\sigma(\o{t}) = N(\alpha(\o{t})) = \partial(\gamma(t))$
    \end{itemize}
  \item All the decompositions I've drawn were tight, even the stupid
    ones.
  \item ``Don't just put extra stuff in the separator''?
  \item Maybe more relevant in graphs with multiple conn.\ comp.'s
    \pause
  \item Lemma 4.4.2: We can make a decomposition tight, but we don't 
    have the close correspondence between nodes of prev. lemmas.
    \begin{itemize}
    \item Among other properties, torsos of the tight decomp.\ are
      subgraphs of the torsos of the original decomp.
    \end{itemize}
    \pause
  \item Lemma 4.4.3 describes a condition under which we can perform a
    decomposition with the same torsos.
    \begin{itemize}
    \item ``\dots will not be needed until late in the second part of the book.
      The reader may consider skipping the (unexpectedly difficult) proof
      for now \dots''
    \end{itemize}
  \end{itemize}
\end{frame}

\begin{frame}
  \frametitle{Tree Decompositions vs.\ Treelike Decompositions}
  Tree Decompositions satisfy TL.1--TL.4.
  \begin{itemize}
  \item \textbf{(T.0)}: $E$ forms a directed tree.
  \item \textbf{T.1}: For all $x$, the set $\{ \overline{t} \st
    \beta(\overline{t}, x)\}$ forms a subtree under $E$.
  \item \textbf{T.2}: For all $x$, $y$ with $E(x, y)$ (in original graph)
    there exists $\overline{t}$ s.t. $\sigma(\overline{t}, x) \And
    \sigma(\overline{t}, y)$.
  \end{itemize}
  \pause
  If $G$ is connected, then any tree decomposition of it is also a treelike decomposition.
\end{frame}

\begin{frame}
  \frametitle{Things in Ch.\ 4 + Ch.\ 5 we haven't covered}
  \begin{itemize}
  \item Homomorphisms and bisimulation (p. 102)
    \pause
  \item We can normalize and use the lifting lemma on parametrized
    decompositions (p. 113-117)
    \pause
  \item Transitivity Lemma (p. 117$+$): For classes $\mathcal{A}$,
    $\mathcal{B}$, $\mathcal{C}$, if $\mathcal{C}$ is IFP-definable over
    $\mathcal{B}$ and $\mathcal{B}$ is IFP-definable over $\mathcal{A}$
    then $\mathcal{C}$ is IFP-definable over $\mathcal{A}$.
    \begin{itemize}
    \item Follows easily from ``Decomposition Lifting Lemma'' (p.118-128)
      \begin{itemize}
      \item Whose proof is ``long and tedious, but in its style typical for
        many proofs of lifting or extension lemmas later in this book
        (mainly in Part II)''  
      \end{itemize}
    \end{itemize}
  \end{itemize}
\end{frame}

% \begin{frame}
% \frametitle{An attempt at tight tree decompositions}
% \includegraphics[scale=0.4]{fig4.pdf}

% An old tree decomposition.
% \end{frame}

% \begin{frame}
% \frametitle{An attempt at tight tree decompositions}
% \includegraphics[scale=0.4]{fig4b.pdf}

% Tight tree decomposition.
% \end{frame}



% If we're moving the separator across the graph, we seem to have skipped a
% few steps at the beginning and end.
% \end{frame}

% \begin{frame}
%     \includegraphics[scale=0.30]{fig5.pdf}
% \end{frame}



\end{document}