\documentclass[12pt]{article}
\usepackage{amssymb,amsfonts,amsmath} % Typical maths resource packages
\usepackage{mathdots}
\usepackage{graphicx}                 % Packages to allow inclusion of graphics
\usepackage{color}                    % For creating coloured text and background
\usepackage{hyperref}                 % For creating hyperlinks in cross references
\usepackage{times}
\usepackage{mydefs}                   % mydefs.sty is my standard tex macros
\longwide
\begin{document}

$[n]$ is the set $\{1, 2, \dots, n\}$.

$\mathopen| G \mathclose|$ is the vertex set of $G$, $\mathopen\| G \mathclose\|$ is the edge set of $G$.

For graph $G$ and set $W \subseteq \mathopen| G \mathclose|$, $G[W]$ is the
induced subgraph of $G$ with vertex set $W$.

$v \unlhd^T w \qE  v$ is an ancestor of $w$.  The root is the min wrt $\unlhd$.

${\cal T}_k$ the set of all graphs of tree width at most $k$.

${\cal K}[V]$ is the complete graph on vertex set $V$.

$K_n$ is ${\cal K}\bigl[[n]\bigr]$ (the complete graph with $n$ vertices).

${\cal K}_n$ is the set of complete graphs with at most $n$ vertices.

${\cal T(C)}$ is the class of graphs having a tree decomposition whose torso is in ${\cal C}$.

${\cal G}_k$ is the class of graphs on at most $k$ vertices.  Thus ${\cal T}_k = {\cal T}({\cal
  G}_{k+1})$.
\end{document}
